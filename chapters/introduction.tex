\chapter{Introduction}

\section{Purpose}
\acrshort{csaf} is a middleware environment for describing, simulating, and analyzing closed loop dynamical systems. A \acrshort{csaf} component's primary interface is subscribe/publish to sockets, which represent input/output of a lump abstracted component. It offers common system representations, especially for specifying controllers. The architecture is supportive of systems to be simulated with diverse components, agnostic to the languages and platforms from which they were implemented.


\section{Document Conventions}
$<$Describe any standards or typographical conventions that were followed when 
writing this SRS, such as fonts or highlighting that have special significance.  
For example, state whether priorities  for higher-level requirements are assumed 
to be inherited by detailed requirements, or whether every requirement statement 
is to have its own priority.$>$


\section{Intended Audience and Reading Suggestions}

This document is designed for planning \acrshort{csaf} features and its use in the \acrlong{aa} project. Relevant team members are intended to read, review, and request changes to the planned architecture and scope. The material is intended to benefit the \acrshort{csaf} contributors and control designers using the project for the \acrlong{aa} project, providing interface descriptions and conveniences to design components following a needed representation.\\

\section{Project Scope}
Its primary use is modeling systems that utilize shields with learning enable controllers (\acrshort{lec}s). Parts of \acrshort{csaf} need to be aware of controller representation; for example, some analyses require explicit knowledge of the fuzzy inference being used. There is value in knowing whether a controller utilizes a linear control law, is predicted by a network architecture and so on.\\

Another use of \acrshort{csaf} is in multi-loop systems, where the control vector gets transformed for entering a plant. For the challenge problem being considered, this separates problems of maneuverability and attitude control. Because of this, exogenic and environmental description is pertinent. Path planning, collision avoidance and environmental sensing are all areas that \acrshort{csaf} will handle.\\

Some of the system description might be done in the \acrshort{edsl} CoPilot, which outputs shared object library. As such, same of the components might be present only as binaries, which need to be utilized for monitoring/controlling. Further, the platform is required for projects where controllers are delivered by other entities, and so agnosticism to language/library/platform is desirable. 

\section{References}

This document uses the \acrshort{ieee} 830 SRS standard, using a \href{https://github.com/jpeisenbarth/SRS-Tex}{\LaTeX{} template} provided on GitHub.  \\
