\chapter{External Interface Requirements}

\section{User Interfaces}

The \acrshort{csaf}  package exists as middleware. It exists as a library, and no CLI or GUI tools are planned.

\section{Software Interfaces}

\subsection{Mathematical Representations}
A component is required to be a dynamical system in some form, a mathematical entity that has the ability to 
evolve in time. This means that a system assumes a function in a set $\left\{ f^t \right\}_{t \in \mathbb F}$, 
where $\mathbb F$ is some set for time. For continuous time systems, this set can be the real numbers $
\mathbb R$; for discrete, the integers $\mathbb Z$ suffices. For systems that use both discrete and 
continuous time, a relation between the two fields is necessary. A sampling frequency and sampling phase is 
used if the discrete time system is uniform time step. This general form, a system as a set of functions, is not 
conducive towards using in a software system, and another representation is desired. \\


Commonly, this set of functions can be characterized by a system of ordinary differential equations 
(\acrshort{ode}s) for continuous time or ordinary difference equations for discrete time. These systems of 
equations can be put into a form that permit a simple representation. For continuous time,

\begin{equation}
\begin{cases}
\dot x = g(t, x, u; p) \\
f_{x_0}^t = h(., x(.), u; r) \\
\end{cases}
\end{equation}

This form separates the solution into two steps with respect to an initial condition $x_0$. First, all variables 
where the \acrshort{ode}s involve rates of change are grouped in a vector $x \in \mathbb R^N$, where it 
assumes an element in a state space. $u \in \mathbb R^M$ is an additional time varying vector, representing 
control input to the system. $p$ and $r$ are time independent parameters involved with the two functions. 
Then, another function $h$ takes the space and returns the system configuration, evaluating one of the 
system functions. Similarly, a construction can be made for a discrete system,

\begin{equation}
\begin{cases}
x_{n} = g(t, n-1, u; p)\\
f^{n}_{x_0} = h(., x(.), u; r) \\
\end{cases},
\end{equation}

for the discrete time system. The evolution of the state $x$ can be computed directly rather than integrating 
the system of equations defining the state differential. Further, time itself need not be directly expressed, an 
essential parameter for non-autonomous systems. The equivalence holds

\begin{equation} \label{equ:cequ}
\begin{bmatrix}
\dot x \\
t
\end{bmatrix} = 
\begin{bmatrix}
g(t, x, u; p) \\
1
\end{bmatrix} \iff \dot x = g(t, x, u;p) .
\end{equation}

However, for the sake of implementation, it is advantageous to express the time parameter directly. To use 
this form, all \acrshort{csaf}  components require two functions, 

\begin{equation} \label{equ:dequ}
\begin{aligned}
g(.,.,.;p):& \mathbb F \times \mathbb R^N \times \mathbb R^M \rightarrow \mathbb R^N \\
h(.,.,.;r):& \mathbb F \times \mathbb R^N \times \mathbb R^M \rightarrow \mathbb R^D
\end{aligned}.
\end{equation}

For continuous time solution of a single component, the function $g(.,.,u(.);p)$ is of a form that common 
\acrshort{ode} solvers can evaluate. The parameters $r$ and $p$ don't necessarily need to be elements of a 
vector space, and can be described using other structures.

\subsubsection{Linear System Representation}

A linear system can take the form,

\begin{equation} \label{equ:linsys}
\begin{cases}
\dot{x} = A x + B u \\
y =C x + D u \\
\end{cases},
\end{equation}

where $x$ is in the state space and $u$ is in the input space. $A$, $B$, $C$, and $D$ are matrices 
representing linear transformations.   \\

For a controlled system, the forcing input $u$ can be the result of a control policy. A control policy itself can
be dependent of the state vector $x$, as is the case for a state space controller. A linear controller follows a
linear control policy,

\begin{equation}
u = K (x_d -x ) = -Ke,
\end{equation}

where $K$ is some linear operator and $e=x -x_d$ represents an error signal from a desired state $x_d$. This 
representation is commonly used, and controller design methods like \acrshort{lqr} and \acrshort{hinf} 
methods, produce an operator $K$ that optimizes some notion of cost. From this view, it is clear that the 
controller is itself a linear system, possessing no state and an input vector $e$. To match the linear 
representation in Equation \ref{equ:linsys},

\begin{equation}
\begin{cases}
\dot{w} = \mathbf{0} w + e \\
u = -K e + G w \\
\end{cases}
\end{equation}

The effect of $w$ is to integrate $e$ with respect to time. This system can represent controllers with tracking
properties, like a \acrshort{pid} controller.

\subsubsection{Neural Controller Representation}\label{sec:nnc}

Neural networks can be used to produce the functions $g$ and $h$ of a controller system, being trained 
beforehand and operating feedforward during use. In this, no distinction is made is made in the representation. \\

\subsubsection{Fuzzy Logic Controller Representation}\label{sec:flc}

A fuzzy logic controller is a controller that uses a \acrlong{fls} to determine its output. Generally, this involves 
taking the input and transforming them to linguistic variables $(x_i, x_2, ..., x_{n_i}) \in \mathbb U^{n_i}$. Then, 
an inference rule $\mathbb R^j$ can be applied from a rules base to produce a linguistic variable describing 
the controller output, being a non-linear mapping between two fuzzy sets $\mathbb U \rightarrow \mathbb R$. 
\begin{equation}
\mathbb{R}^{(j)}:\text{ IF }x_{1}\text{ is }A_{1}^{j}\text{ and }\ldots\text{ and }x_{n i}\text{ is }A_{n i}^{j}\text{ THEN }y\text{ is }B^{j}, j=1, \ldots, n_{j},
\end{equation}
where $A_i^j$, $B^j$ are fuzzy sets in $\mathbb U$ and $\mathbb R$. This set can be ``defuzzified'' into a crisp value $y \in \mathbb R$ to produce a control signal. \\

A controller adhering to this paradigm is stateless, and need only implement the output mapping $h$
mentioned previously that performs the fuzzification, inference, and defuzzification. Besides this, the fuzzy 
sets that the input and output assume, the fuzzication/defuzzification startegies, and the inference rules employed, are necessary for the fuzzy logic controller representation.

\subsection{Model Interfaces}

A concept of a model is derived from the representations in \ref{equ:cequ} and \ref{equ:dequ}. Figure \ref{fig:mio} presents the IO relationships of a \acrshort{csaf} model. The function arguments serve as the input, and evaluations of $g$ and $h$ can be the output.  A third output, labeled ``info'' is available to pass specific information about the model representation (such as the ones seen in \ref{sec:flc} and \ref{sec:nnc}). Note that the model does not use any time varying parameter as a state, requiring all such quantities having to be passed manually. The state of a dynamical system does not serve as a state represented by the programmatic object model. Time invariant parameters are states in the object and settable. 

\begin{figure}
\centering
\includegraphics[width=1.0\textwidth]{model.pdf}
\caption{\acrshort{csaf} Model IO. A model requires time, state, and input to be passed in order to collect output. A model can have associated parameters that can be passed in once and set.}
\label{fig:mio}
\end{figure}

\subsection{Component Interfaces}

A component is an object that presents a model as an element of a publish/subscribe architecture, with it IO diagram presented in Figure \ref{fig:cio}. It subscribes to topics received by its input sockets. For normal inputs, the messages received are deserialized into a control signal that is stored into an input buffer. This buffer stores the necessary members to update a system represented by a \acrshort{csaf} model object. The model output is serialized into a collection of output messages, and then send over a single output socket.\\

An events socket is also required (input c); messages can be sent in order to configure the component internals. These actions are

\begin{enumerate}
\item reset the buffers to default
\item configure buffer defaults
\item enable debugging
\item pause/resume the component
\item update the component
\item delete the component, unbinding/disconnecting any ports
\end{enumerate}

\begin{figure}
\centering
\includegraphics[width=1.0\textwidth]{componentio.pdf}
\caption{\acrshort{csaf}  Component IO. For input, sockets receive a message with specific topics. The output is a single socket, producing messages with different topics.}
\label{fig:cio}
\end{figure}

\subsection{System Description/Initialization}

Once valid components are made, they can be composed together to make systems. This composition can be described in a TOML file (Code Block \ref{fig:sysset}). 

\begin{figure}
\begin{lstlisting}
name = "inverted-pendulum"

# directory setup
codec_dir = "codec"
output_dir = "output"

# order to construct/evaluate components 
evaluation_order = ["maneuver", "controller", "plant"]

# log setup
log_file = "inv_pendulum.log"
log_level = "info"

# port to broadcast events to the components
events_port = 6001

[components]

  [components.controller]
    # environment to run under
    run_command = "python3"

    # path to model executable
    process = "ipcontroller.py"

    # whether to print debug diagnostics
    debug = false

    # subscribe to topic of a component (component name, topic name)
    sub = [["plant", "states"], ["maneuver", "outputs"]]

    # port to publish
    pub = 5502

  [components.plant]
    run_command = "python3"
    process = "ipplant.py"
    debug = false
    sub = [["controller", "outputs"]]
    pub = 5501

  [components.maneuver]
    run_command = "python3"
    process = "ipmaneuver.py"
    debug = false
    sub = []
    pub = 5503
\end{lstlisting}
\caption{Example TOML File Describing Components and System Compositions of Components}
\label{fig:sysset}
\end{figure}

\section{Communications Interfaces}

Analyzers and simulators will interact with the system components via \acrshort{0mq}, using sockets. \acrshort{0mq} 
provides \acrshort{csaf}  with a variety of levels to communicate with components, whether in-process, 
inter-process or across TCP and multi-cast. The component does hold state, and the system needs to be 
properly initialized to avoid staleness.

\subsection{Message Contents}

The temporal messages are sent by every \acrshort{csaf}  component that represent vectors. The message type is derived from the \acrshort{ros}msgs serialization format. First, the the 
\acrshort{csaf}  version is transmitted to avoid incompatibility between components made with different 
versions. Second, system time is included to ensure that components evaluate  
correctly. Next is the vector. These vectors are enumerated under a header, rather than transmitted as a 
contiguous array. Figure \ref{fig:cmsg} shows an example message. \\

Non-temporal aspects of the system need to need to be accessed, which is described in a configuration file associated with a component. The system name, parameters, representation identifier and solver name is included. Two booleans are used to determine if the system type---continuous, discrete or hybrid.  Fields specific to the representation of the system is  accessible. For example, if the system is a fuzzy controller, the inference table can be visible. As the  components have a varying degree of transparency (``black box''), no headers are required and parameters are allowed to unchangeable. \\

A controller's representation and purpose influences its requirements. As such, much 
variability can exist in what is contained in the message. New headers and variables are allowed to appear in a 
component message, but the structure has to remain constant over time. \\

\begin{figure}
\begin{lstlisting}
uint32 version_major
uint32 version_minor

string topic

float64 time
\end{lstlisting}
\caption{Required \acrshort{ros}msg \acrshort{csaf}  Component Temporal Message Contents (Other Fields are Permitted)}
\label{fig:cmsg}
\end{figure}

\begin{figure}
\begin{lstlisting}
system_name = "Inverted Pendulum Plant"
system_representation = "black box"
system_solver = "Euler"

sampling_frequency = 100

is_discrete = false
is_hybrid = false

[parameters]
  mm = 0.5        # Mass of the cart
  m = 0.2         # Mass of the pendulum
  b = 0.1         # coefficient of friction on cart
  ii = 0.006      # Mass Moment of Inertia on Pendulum
  g = 9.8         # Gravitational acceleration
  length = 0.3    # length of pendulum

[inputs]
  msgs = ["ipcontroller_output.msg"]

[topics]

  [topics.states]
    msg = "ipplant_state.msg"
    initial = [0.0, 0.01, 0.52 , -0.01]
\end{lstlisting}
\caption{ \acrshort{csaf}  Component Configuration Information)}
\label{fig:cmsg}
\end{figure}

